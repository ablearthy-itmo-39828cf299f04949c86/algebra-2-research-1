\subsection{Задача № 3Б}

Дано пространство функций \(L\)
--- пространство многочленов степени не выше второй
с вещественными коэффициентами,
и отображение \(\opA : L \to L\):
\[
  \opA f = f'' - 3 f' + f
\]
и вектор \(p(t) = 3t^2 + t + 2\)

\subsubsection{Решение}

Выберем базис для \(L\).

Выберем базис \(E = \{e_1 = 1, e_2 = t, e_3 = t^2\}\)
Докажем, что любой многочлен степени не выше второй можно представить
как линейную комбинацию системы \(E\).
Возьмем произвольный многочлен из \(L\): \(q = a + bt + ct^2\).
Составим линейную комбинацию из \(E\):
\(\Lambda = \alpha + \beta t + \gamma t^2\).
Многочлены равны, когда \(\alpha = a, \beta = b, \gamma = c\).
Так как \(a, b, c\) произвольные,
любой многочлен можно представить в виде линейной комбинации \(E\).

Покажем, что \(E\) линейно независимая система.
Рассмотрим линейную комбинацию: $\alpha + \beta t + \gamma t^2 = 0$.
Тогда по определению равенства многочленов ---
многочлены \(f(x)\) и \(g(x)\) называются равными, если
имеют одинаковые степени и, кроме того, их коэффициенты при
аковых степенях переменной x равны между собой ---
\(\alpha = \beta = \gamma = 0\).
Значит, линейная комбинация тривиальна, а система \(E\) является базисом.

Убедимся, что отображение \(\opA\) является линейным.

Отображение \(\opA: L \to L\) будет являться линейным оператором,
если \(\forall f,g \in L, \lambda\in\mathbb{R}\):
\begin{align*}
  \opA(f + g) &= \opA(f) + \opA(g) \\
  \opA(λ f) &= λ\opA(f)
\end{align*}

Проверим это.
\begin{align*}
  \opA(f + g) &= (f + g)'' - 3 (f + g)' + (f + g) \\
              &= (f'' - 3f' + f) + (g'' - 3g' + g) = \opA(f) + \opA(g) \\
  \opA(λ f) &= (λf)'' - 3 (λf)' + λf = λ(f'' - 3f' + f) = λ \opA(f)
\end{align*}

В самом деле, \(\opA\) является линейным.

Найдём матрицу оператора \(\opA\) в базисе \(E\) и его ранг.
\begin{align*}
  \opA(e_1) &= \opA(1) = 1 = 0 t + 0 t^2 \\
  \opA(e_2) &= \opA(t) = -3 + t + 0 t^2 \\
  \opA(e_3) &= \opA(t^2) = 2 - 6t + t^2
\end{align*}


Составим матрицу оператора:
\[
  \matA_e
  =
  \begin{pmatrix}
      1 & -3 & 2\\
      0 & 1 & -6\\
      0 & 0 & 1
  \end{pmatrix}
\]

Найдём размерности ядра и образа оператора \(\opA\).

Сначала найдем ядро оператора.
По определению: \(\ker\opA = \{x \mid \mathcal{A}x = 0\}\).
Возьмем произвольный \(x = at^2 + bt + c\).
\begin{align*}
  \opA(x) &= 0 \\
  \opA(at^2 + bt + c) &= 0 \\
  {(at^2 + bt + c)}^{\prime\prime}
  - 3{(at^2 + bt + c)}^{\prime}
  + (at^2 + bt + c) &= 0 \\
  2a - 6at - 3b + at^2 + bt + c &= 0 \\
  at^2  + (-6a + b)t + (2a - 3b + c) &= 0 \\
  \begin{cases}
    a = 0\\
    -6a + b = 0\\
    2a - 3b + c = 0
  \end{cases}
  &
  \begin{cases}
    a = 0\\
    b = 0\\
    c = 0
  \end{cases}
\end{align*}
Получили, что \(\ker \opA = \{\vec{0}\}\).
Значит \(\dim(\ker \opA) = 0\).
Для того чтобы найти размерность образа,
воспользуемся теоремой о сумме размерности ядра и образа оператора:
\[\dim(\ker \opA) + \dim(\mIm \opA) = \dim(L)\]
Так как выбранный базис состоит из трех векторов, \(\dim(L) = 3\). Значит,
\(\dim(\mIm \opA) = 3\)


Найдём собственные числа и векторы оператора.
Определим размерность пространства собственных векторов.

Найдем собственные числа:
\begin{align*}
  \det(\matA - \lambda \mathrm{E}) &= 0 \implies
  \begin{vmatrix}
      1 - λ & -3 & 2 \\
      0 & 1 - λ & -6 \\
      0 & 0 & 1 - λ
  \end{vmatrix} = 0 \\
  (1 - \lambda)^3 &= 0 \\
  \lambda_{1,2,3} &= 1
\end{align*}

Найдем собственные векторы:
\begin{align*}
  &
  \left(
    \begin{matrix}
        1 - 1 & -3 & 2 \\
        0 & 1 - 1 & -6 \\
        0 & 0 & 1 - 1 \\
    \end{matrix}
  \right|
  \left. \begin{matrix} 0 \\ 0 \\ 0 \end{matrix} \right)
  \rightsquigarrow
  \left(
    \begin{matrix}
        0 & 3 & -2 \\
        0 & 0 & 1 \\
    \end{matrix}
  \right|
  \left. \begin{matrix}  0 \\ 0 \end{matrix} \right) \\
  & \rightsquigarrow
  \left(
    \begin{matrix}
        0 & 1 & 0 \\
        0 & 0 & 1 \\
    \end{matrix}
  \right|
  \left. \begin{matrix}  0 \\ 0 \end{matrix} \right)
  \implies
  \begin{cases}
    x_1 = α \\
    x_2 = 0 \\
    x_3 = 0
  \end{cases}
\end{align*}

Получили собственный вектор:
\[x = \begin{pmatrix} 1\\0\\0  \end{pmatrix}\]

Размерность пространства собственных векторов:
\[\dim(\mSpan\{x\}) = 1\]

Вывод: если размерность пространства собственных векторов
равна размерности матрицы оператора,
то эту матрицу можно диагонализировать при переходе в базис,
который образовывают собственные векторы,
при этом на диагонали будут находиться собственные числа оператора.

Найдём \(p(t)\) умножением на матрицу оператора.
Проверим результат дифференцированием.
\[
  \begin{pmatrix}
    1 & -3 & 2\\
    0 & 1 & -6\\
    0 & 0 & 1
  \end{pmatrix}
  \begin{pmatrix} 2\\ 1\\ 3 \end{pmatrix}
  =
  \begin{pmatrix}
      2 - 3 + 6 \\
      0 + 1 - 18 \\
      0 + 0 + 3
  \end{pmatrix}
  =
  \begin{pmatrix} 5 \\ -17 \\ 3 \end{pmatrix}
\]

\[
\begin{split}
  \opA(p(t))
  = (2 + t + 3t^2)^{\prime\prime}
  - 3(2 + t + 3t^2)^\prime
  + (2 + t + 3t^2)
  = 6 - 3(1 + 6t) + (2 + t + 3t^2) \\
  = 6 - 3 - 18t + 2 + t + 3t^2
  = 5 - 17t + 3t^2
\end{split}
\]

Результаты совпали.
