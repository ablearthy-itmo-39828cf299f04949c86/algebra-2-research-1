\section{Задание № 2}

Дано уравнение:
\[2x^2 - 6xy + 2y^2 + z^2 - 25 = 0\]

\begin{enumerate}
  \item Составьте матрицу квадратичной формы и диагонализируйте ее.
    Запишите канонический базис квадратичной формы.
  \item Классифицируйте поверхность по ее каноническому уравнению.
  \item Определите, каким преобразованием пространства поверхность была
    приведена к главным осям.
  \item Изобразите график уравнения в исходной системе координат.
    Укажите на графике оси исходной и приведённой систем координат
\end{enumerate}

\subsection{Решение}

\begin{figure}[htbp]
  \centering
  \asyinclude{asy/02.asy}
  \caption{Графическая иллюстрация к задаче № 2} \label{fig:2}
\end{figure}


Составим матрицу квадратичной формы.

\[
  2x^2 + 2y^2 + z^2
  + 2 (-3) xy + 2 \cdot 0 \cdot xz + 2 \cdot 0 \cdot yz
  + 2 \cdot 0 \cdot y + 2 \cdot 0 \cdot z + (-25) = 0
  \Rightarrow
  A =
  \begin{pmatrix}
    2 & -3 & 0 \\
    -3 & 2 & 1 \\
    0 & 0 & 1
  \end{pmatrix}
\]

Диагонализируем её. Для этого найдём собственные числа
и собственные векторы.
\[
  \begin{vmatrix}
    2 - λ & -3 & 0 \\
    -3 & 2 - λ & 1 \\
    0 & 0 & 1 - λ
  \end{vmatrix}
  = 0
  \Leftrightarrow
  (1 - λ) ({(2 - λ)}^2 - (-3) (-3)) = 0
  \Leftrightarrow
  (1 - λ) ({(2 - λ)}^2 - 9) = 0
\]
Отсюда получаем, что
\(λ_1 = 1\),
\(λ_2 = -1\),
\(λ_3 = 5\).
Тогда диагональная матрица такова:
\[
  Λ =
  \begin{pmatrix}
    -1 & 0 & 0 \\
    0 & 1 & 0 \\
    0 & 0 & 5
  \end{pmatrix}
\]

Найдём собственные векторы:
\begin{itemize}
\item для \(λ_1\):
\[
  \begin{pmatrix}
    2 + 1 & -3 & 0 \\
    -3 & 2 + 1 & 0 \\
    0 & 0 & 1 + 1
  \end{pmatrix} \cdot
  \begin{pmatrix}
    x \\
    y \\
    z
  \end{pmatrix}
  = 0
  \implies
  l_1 =
  \begin{pmatrix}
    x \\
    y \\
    z
  \end{pmatrix} = α
  \begin{pmatrix}
    1 \\
    1 \\
    0
  \end{pmatrix}
\]
\item для \(λ_2\):
\[
  \begin{pmatrix}
    2 - 1 & -3 & 0 \\
    -3 & 2 - 1 & 0 \\
    0 & 0 & 1 - 1
  \end{pmatrix} \cdot
  \begin{pmatrix}
    x \\
    y \\
    z
  \end{pmatrix}
  = 0
  \implies
  l_2 =
  \begin{pmatrix}
    x \\
    y \\
    z
  \end{pmatrix} = β
  \begin{pmatrix}
    0 \\
    0 \\
    1
  \end{pmatrix}
\]
\item для \(λ_3\):
\[
  \begin{pmatrix}
    2 - 5 & -3 & 0 \\
    -3 & 2 - 5 & 0 \\
    0 & 0 & 1 - 5
  \end{pmatrix} \cdot
  \begin{pmatrix}
    x \\
    y \\
    z
  \end{pmatrix}
  = 0
  \implies
  l_3 =
  \begin{pmatrix}
    x \\
    y \\
    z
  \end{pmatrix} = γ
  \begin{pmatrix}
    1 \\
    -1 \\
    0
  \end{pmatrix}
\]
\end{itemize}
Пусть:
\[
  l_1 =
  \begin{pmatrix}
    1 \\
    1 \\
    0
  \end{pmatrix} \quad
  l_2 =
  \begin{pmatrix}
    0 \\
    0 \\
    1
  \end{pmatrix} \quad
  l_3 =
  \begin{pmatrix}
    1 \\
    -1 \\
    0
  \end{pmatrix}
\]
Собственные векторы попарно ортогональны,
нормируем их:
\[
  s_1 = \frac{l_1}{\sqrt{1 + 1}} =
  \begin{pmatrix}
    \frac{1}{\sqrt{2}} \\
    \frac{1}{\sqrt{2}} \\
    0
  \end{pmatrix} \quad
  s_2 = l_2 =
  \begin{pmatrix}
    0 \\
    0 \\
    1
  \end{pmatrix} \quad
  s_3 = \frac{l_3}{\sqrt{1 + 1}} =
  \begin{pmatrix}
    \frac{1}{\sqrt{2}} \\
    -\frac{1}{\sqrt{2}} \\
    0
  \end{pmatrix}
\]
Следовательно,
\[
  S =
  \begin{pmatrix}
    \frac{1}{\sqrt{2}} & 0 & \frac{1}{\sqrt{2}} \\
    \frac{1}{\sqrt{2}} & 0 & -\frac{1}{\sqrt{2}} \\
    0 & 1 & 0
  \end{pmatrix}
\]

Члены первого порядка образуют матрицу:
\[
  B =
  \begin{pmatrix}
    0 \\ 0 \\ 0
  \end{pmatrix}
\]
Найдём коэффициенты при них в новом базисе:
\[
  B' = B S = B
\]

Следовательно, канонический вид:
\[-{x'}^2 + {y'}^2 + 5 {z'}^2 - 25 = 0\]
Из чего следует:
\[-\frac{{x'}^2}{25} + \frac{{y'}^2}{25} + \frac{{z'}^2}{5} = 1\]

Итак, это однополостный гиперболоид с центральной осью \(Ox'\).

Переход от новых координат к старым.
\[
  \begin{pmatrix} x \\ y \\ z \end{pmatrix} =
  \begin{pmatrix}
    \frac{1}{\sqrt{2}} & 0 & \frac{1}{\sqrt{2}} \\
    \frac{1}{\sqrt{2}} & 0 & -\frac{1}{\sqrt{2}} \\
    0 & 1 & 0
  \end{pmatrix}
  \begin{pmatrix}
    x' \\ y' \\ z'
  \end{pmatrix}
\]

Поверхность была приведена к главным осям посредством двух преобразований:
\begin{enumerate}
  \item вращением поверхности вокруг \(Oz\) на \(45^{\circ}\);
  \item вращением поверхности вокруг \(Ox\) на \(90^{\circ}\).
\end{enumerate}
