\subsection{Задание № 1 Б}

Дано пространство \(R\) функций, непрерывных на отрезке \([-\pi; \pi]\)
со скалярным произведением
\(\dotprod{f, g} = \int_{-\pi}^{\pi} f(t) g(t) \dd t \)
и длиной вектора \(||f|| = \sqrt{\dotprod{f, f}}\).

Тригонометрические многочлены
\[
  P_{n}(t) = \frac{a_0}{2} + a_1 \cos(t) + b_1 \sin(t) + \ldots
  + a_n \cos(n t) + b_n \sin(n t)
\]
где \(a_k, b_k\) --- вещественные коэффициенты,
образующие подпространство \(P\) подпространства \(R\).
Требуется найти многочлен \(P_{n}(t)\) в пространстве \(P\),
минимально отличающийся от функции \(f(t) = 2t\)
--- вектора пространства \(R\).

\subsubsection{Решение}

Проверим, что система функций
\(\{1, \cos(t), \sin(t), \ldots \cos(nt), \sin(nt)\}\)
является ортогональным базисом подпространства \(P\).

Эта система является базисом подпространства \(P\)
по его определению.
Осталось проверить ортогональность этой системы, для этого посчитаем
скалярные произведения элементов \(n \neq m\):
\[
  \dotprod{1, \cos nx} = \int_{-\pi}^{\pi} \cos nx \dd x
  = \frac{\sin nx}{n} \bigg\vert_{-\pi}^{\pi}
  = \frac{\sin(\pi n) - \sin(-\pi n)}{n} = 0
\]
\[
  \dotprod{1, \sin nx} = \int_{-\pi}^{\pi} \sin nx \dd x
  = -\frac{\cos nx}{n} \bigg\vert_{-\pi}^{\pi}
  = \frac{-\cos(\pi n) + \cos(-\pi n)}{n} = 0
\]
\[
\begin{split}
  \dotprod{\cos nx, \cos mx}
  &= \int_{-\pi}^{\pi} \cos nx \cos mx \dd x \\
  &= \frac{1}{2}
    \int_{-\pi}^{\pi} (\cos((n - m) x) + \cos((n + m) x)) \dd x \\
  &= \frac{1}{2} \left.\left(
      \frac{\sin((n - m) x)}{n - m} + \frac{\sin((n + m) x)}{n + m}
  \right) \right\vert_{-\pi}^{\pi} = 0
\end{split}
\]
\[
\begin{split}
  \dotprod{\sin nx, \sin mx}
  &= \int_{-\pi}^{\pi} \sin nx \sin mx \dd x \\
  &= \frac{1}{2} \int_{-\pi}^{\pi} (\cos((n - m) x) - \cos((n + m) x)) \dd x \\
  &= \frac{1}{2} \left.\left(
      \frac{\sin((n - m) x)}{n - m} - \frac{\sin((n + m) x)}{n + m}
  \right) \right\vert_{-\pi}^{\pi} = 0
\end{split}
\]
\[
\begin{split}
  \dotprod{\sin nx, \cos mx}
  &= \int_{-\pi}^{\pi} \sin nx \cos mx \dd x \\
  &= \frac{1}{2} \int_{-\pi}^{\pi} (\sin((n - m) x) + \sin((n + m) x)) \dd x
\end{split}
\]
Данный интеграл равен нулю по свойству определённого интеграла
от нечётных функций на симметричном отрезке. Следовательно,
\[\dotprod{\sin nx, \cos mx} = 0\]
Нормируем эту систему, для этого вычислим нормы элементов:
\[
  \dotprod{1,1}
  = \int_{-\pi}^{\pi} 1 \dd x
  = 2\pi
  \implies
  \lVert 1 \rVert
  = \sqrt{2 \pi}
\]
\[
\begin{split}
\dotprod{\cos nx, \cos nx}
  &= \int_{-\pi}^{\pi} {(\cos nx)}^2 \dd x
  = \frac{1}{2} \int_{-\pi}^{\pi} (1 + \cos(2nx)) \dd x \\
  &= \frac{1}{2} \left.
    \left( x + \frac{\sin(2nx)}{2n} \right)
    \right\rvert_{-\pi}^{\pi} = \pi
  \implies \lVert \cos nx \rVert = \sqrt{\pi}
\end{split}
\]
\[
\begin{split}
  \dotprod{\sin nx, \sin nx}
  &= \int_{-\pi}^{\pi} {(\sin nx)}^2 \dd x
  = \frac{1}{2} \int_{-\pi}^{\pi} (1 - \cos(2nx)) \dd x \\
  &= \frac{1}{2} \left.
    \left( x - \frac{\sin(2nx)}{2n} \right)
    \right\rvert_{-\pi}^{\pi} = \pi
  \implies \lVert \sin nx \rVert = \sqrt{\pi}
\end{split}
\]
Значит ортонормированная система будет иметь вид:
\[
  \left\{
    \frac{1}{\sqrt{2 \pi}},
    \frac{\cos t}{\sqrt{\pi}},
    \frac{\sin t}{\sqrt{\pi}},
    \ldots
    \frac{\cos nt}{\sqrt{\pi}},
    \frac{\sin nt}{\sqrt{\pi}}
  \right\}
\]

Найдём проекции вектора \(f(t) = 2t\) на векторы полученного
ортонормированного базиса.
Процесс вычисления описан в приложении \ref{app:b-1-calc}
\begin{align*}
  \dotprod{\frac{1}{\sqrt{2\pi}}, 2t} &= 0 \\
  \dotprod{\frac{\cos nt}{\sqrt{\pi}}, 2t} &= 0 \\
  \dotprod{\frac{\sin nt}{\sqrt{\pi}}, 2t}
    &= -\frac{4 \sqrt{\pi} \cos(\pi n)}{n}
\end{align*}

Так как базис ортонормированный,
то полученные координаты есть суть проекции
вектора \(f(t)\) на базисные векторы.

Запишем минимально остоящий многочлен \(P_{n}(t)\)
с найденными коэффициентами:
\[
  P_{n}(t)
  = \sum_{n = 1}^{\infty}
    \left(-\frac{4\cos(\pi n)}{n} \sin(nt)\right)
  = 4 \sum_{n = 1}^{\infty}
    \left({(-1)}^{n + 1} \frac{\sin(nt)}{n}\right)
\]
Изобразим графики функций \(f(t)\) и многочлена Фурье
различных порядков \(n\).

\begin{figure}[!htbp]
  \centering
  \begin{subfigure}{0.32\textwidth}
    \centering
    \asyinclude{asy/01-b-01.asy}
    \caption{\(n = 5\)}
  \end{subfigure}
  \begin{subfigure}{0.32\textwidth}
    \centering
    \asyinclude{asy/01-b-02.asy}
    \caption{\(n = 10\)}
  \end{subfigure}
  \begin{subfigure}{0.32\textwidth}
    \centering
    \asyinclude{asy/01-b-03.asy}
    \caption{\(n = 15\)}
  \end{subfigure}
  \caption{Графики функций \(f(x)\) и \(P_{n}(x)\)}
\end{figure}

Таким образом,
при увеличении порядка многочлен все точнее приближает исходную функцию.
