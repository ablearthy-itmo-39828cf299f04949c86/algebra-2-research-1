\subsection{Задание № 1 А}

\newcommand{\Bs}{\mathcal{B}}
\newcommand{\BsH}{{\Bs_H}}

Дано пространство многочленов с вещественными коэффициентами,
степени не выше \textit{третьей}, определенных на отрезке \([-1; 1]\).

Проведите исследование:
\begin{enumerate}
  \item Проверьте, что система векторов \(\Bs = \{1, t, t^2\}\) является
    базисом этого пространства.
    Ортогонализируйте систему, получив ортогональный базис \(\BsH\).
  \item Выпишите первые четыре (\(n \in \{0, 1, 2, 3\}\)) многочлена Лежандра:
    \[
      L_{n}(t)
      = \frac{1}{2^{n} n!} \diff[n]{}{t}\left( {(t^2 - 1)}^{n} \right),
    \]
    где \(\diff[n]{}{t}\left( y(t) \right)\) --- производная \(n\)-го порядка
    функции \(y(t)\).
  \item Найдите координаты полученных многочленов \(L_{n}(t)\) в базисе
    \(\BsH\).
    Сделайте вывод об ортогональности системы векторов \(L_{n}(t)\).
  \item Разложите данный многочлен \(P_{3}(t) = t^3 - 2 t^2 + t + 1\)
    по системе векторов \(L_{n}(t)\).
\end{enumerate}

\subsubsection{Решение}

Для начала проверим, является ли система векторов \(\Bs\) базисом.
Для этого проверим линейную зависимость элементов этой системы.
\[
  \left\{
  \begin{aligned}
    1  &= α_{11} t + α_{12} t^2 &+ α_{13} t^3 \\
    t  &= α_{21} 1 + α_{22} t^2 &+ α_{23} t^3 \\
    t² &= α_{31} 1 + α_{32} t  &+ α_{33} t^3 \\
    t³ &= α_{41} 1 + α_{42} t  &+ α_{43} t^2
  \end{aligned}
  \right.
\]
Ни одно из этих уравнений не имеет решений относительно \(α\), т.к.
в правых частях уравнений нет многочленов той же степени, что и в левой.
Таким образом, система \(\Bs\) линейно независима.

В силу того, что любой многочлен \(P_{3}(x)\) с вещественными коэффициентами,
степени не выше третьей, может быть представлен в виде
\[P_{3}(x) = α 1 + β t + γ t^2 + δ t^3\]
то система \(\Bs\) является базисом этого пространства.

Произведём ортогонализацию данного базиса (используя процесс Грама--Шмидта):
\begin{align*}
  \BsH_1 &= \Bs_1 \\
  \BsH_2 &= \Bs_2 - \myproj{\Bs_2}{\BsH_1} \\
  \BsH_3 &= \Bs_3 - \myproj{\Bs_3}{\BsH_1} - \myproj{\Bs_3}{\BsH_2} \\
  \BsH_4 &= \Bs_4 - \myproj{\Bs_4}{\BsH_1} - \myproj{\Bs_4}{\BsH_2}
    - \myproj{\Bs_4}{\BsH_3}
\end{align*}

В приложении \ref{app:a-1-calc} произведены необходимые вычисления.
Итого:
\[\BsH = \left\{ 1, t, t^2 - \frac{1}{3}, t^3 - \frac{3}{5}t \right\}\]
