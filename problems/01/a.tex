\subsection{Задание № 1 А}

\newcommand{\Bs}{\mathcal{B}}
\newcommand{\BsH}{{\Bs_H}}
\newcommand{\Leg}[1][n]{{L_{#1}(t)}}

Дано пространство многочленов с вещественными коэффициентами,
степени не выше \textit{третьей}, определенных на отрезке \([-1; 1]\).

Проведите исследование:
\begin{enumerate}
  \item Проверьте, что система векторов \(\Bs = \{1, t, t^2\}\) является
    базисом этого пространства.
    Ортогонализируйте систему, получив ортогональный базис \(\BsH\).
  \item Выпишите первые четыре (\(n \in \{0, 1, 2, 3\}\)) многочлена Лежандра:
    \[
      L_{n}(t)
      = \frac{1}{2^{n} n!} \diff[n]{}{t}\left( {(t^2 - 1)}^{n} \right),
    \]
    где \(\diff[n]{}{t}\left( y(t) \right)\) --- производная \(n\)-го порядка
    функции \(y(t)\).
  \item Найдите координаты полученных многочленов \(L_{n}(t)\) в базисе
    \(\BsH\).
    Сделайте вывод об ортогональности системы векторов \(L_{n}(t)\).
  \item Разложите данный многочлен \(P_{3}(t) = t^3 - 2 t^2 + t + 1\)
    по системе векторов \(L_{n}(t)\).
\end{enumerate}

\subsubsection{Решение}

Для начала проверим, является ли система векторов \(\Bs\) базисом.
Для этого проверим линейную зависимость элементов этой системы.
\[
  \left\{
  \begin{aligned}
    1  &= α_{11} t + α_{12} t^2 &+ α_{13} t^3 \\
    t  &= α_{21} 1 + α_{22} t^2 &+ α_{23} t^3 \\
    t² &= α_{31} 1 + α_{32} t  &+ α_{33} t^3 \\
    t³ &= α_{41} 1 + α_{42} t  &+ α_{43} t^2
  \end{aligned}
  \right.
\]
Ни одно из этих уравнений не имеет решений относительно \(α\), т.к.
в правых частях уравнений нет многочленов той же степени, что и в левой.
Таким образом, система \(\Bs\) линейно независима.

В силу того, что любой многочлен \(P_{3}(x)\) с вещественными коэффициентами,
степени не выше третьей, может быть представлен в виде
\[P_{3}(x) = α 1 + β t + γ t^2 + δ t^3\]
то система \(\Bs\) является базисом этого пространства.

Произведём ортогонализацию данного базиса (используя процесс Грама--Шмидта):
\begin{align*}
  \BsH_1 &= \Bs_1 \\
  \BsH_2 &= \Bs_2 - \myproj{\Bs_2}{\BsH_1} \\
  \BsH_3 &= \Bs_3 - \myproj{\Bs_3}{\BsH_1} - \myproj{\Bs_3}{\BsH_2} \\
  \BsH_4 &= \Bs_4 - \myproj{\Bs_4}{\BsH_1} - \myproj{\Bs_4}{\BsH_2}
    - \myproj{\Bs_4}{\BsH_3}
\end{align*}

В приложении \ref{app:a-1-calc} произведены необходимые вычисления.
Итого:
\[\BsH = \left\{ 1, t, t^2 - \frac{1}{3}, t^3 - \frac{3}{5}t \right\}\]

Теперь выпишем первые четыре многочлена Лежандра
\begin{align*}
  L_0(t) &= 1 \\
  L_1(t) &= \frac{1}{2} \cdot 2t = t \\
  L_2(t) &= \frac{1}{8} {\left( 2(t^2 - 1) \cdot 2t \right)}'
    = \frac{1}{8} (12t^2 - 4) = \frac{1}{2} (3t^2 - 1) \\
  L_3(t) &= \frac{1}{48} {\left( 3(t^2 - 1)^2 \cdot 2t \right)}''
    = \frac{1}{48} {(6 t^5 - 12 t^3 + 6t)}''
    = \frac{1}{48} (120 t^3 - 72 t)
    = \frac{1}{2} (5 t^3 - 3 t)
\end{align*}

Найдём координаты полученных многочленов \(\Leg\) в базисе \(\BsH\).
В базисе \(\BsH\):
\begin{itemize}
  \item \(\Leg[0]\) имеет координаты \(\coord{1, 0, 0, 0}\)
  \item \(\Leg[1]\) имеет координаты \(\coord{0, 1, 0, 0}\)
  \item \(\Leg[2]\) имеет координаты \(\coord{0, 0, 3/2, 0}\)
  \item \(\Leg[3]\) имеет координаты \(\coord{0, 0, 0, 5/2}\)
\end{itemize}

Таким образом, каждый из векторов системы \(\Leg\)
получен из некоторого вектора системы \(\BsH\) сжатием/растяжением.
Следовательно из-за ортогональности \(\BsH\), то и система \(\Leg\)
тоже будет ортогональна, ведь растяжение/сжатие не нарушает ортогональности
(из-за свойств скалярного произведения).

Разложим данный многочлен \(P_{3}(t)\) по системе векторов \(\Leg\):
\begin{align*}
  P_{3}(t) &= α \Leg[0] + β \Leg[1] + γ \Leg[2] + δ \Leg[3] \\
  t^3 - 2t^2 + t + 1 &= α + βt + γ \frac{3}{2} t^2 - γ \frac{1}{2}
    + δ \frac{5}{2} t^3 - δ \frac{3}{2} t
\end{align*}
Получаем СЛАУ:
\[
  \left\{
    \setlength\arraycolsep{0pt}
    \begin{array}{r @{{}={}} r  >{{}}c<{{}} r  >{{}}c<{{}}  r >{{}}c<{{}}  r}
      1  & α  & &   &-& \frac{γ}{2} \\
      1  &    &\phantom{+}& β & & &-&\frac{3δ}{2} \\
      -2 &    & &   & & \frac{3γ}{2} \\
      1  &    & &   & & & & \frac{5δ}{2}
    \end{array}
  \right.
\]
Откуда находим:
\begin{align*}
  δ &= \frac{2}{5} \\
  γ &= -\frac{4}{3} \\
  β &= 1 + \frac{3}{2} \cdot \frac{2}{5} = \frac{8}{5} \\
  α &= 1 - \frac{4}{3 \cdot 2} = \frac{1}{3}
\end{align*}

Таким образом, разложение \(P_{3}(t)\) по системе \(\Leg\)
будет иметь вид:
\[
  P_3(t)
  = \frac{1}{3} \Leg[0]
  + \frac{8}{5} \Leg[1]
  - \frac{4}{3} \Leg[2]
  + \frac{2}{5} \Leg[3]
\]
